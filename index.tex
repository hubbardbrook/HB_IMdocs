% Options for packages loaded elsewhere
\PassOptionsToPackage{unicode}{hyperref}
\PassOptionsToPackage{hyphens}{url}
\PassOptionsToPackage{dvipsnames,svgnames,x11names}{xcolor}
%
\documentclass[
  letterpaper,
  DIV=11,
  numbers=noendperiod]{scrreprt}

\usepackage{amsmath,amssymb}
\usepackage{iftex}
\ifPDFTeX
  \usepackage[T1]{fontenc}
  \usepackage[utf8]{inputenc}
  \usepackage{textcomp} % provide euro and other symbols
\else % if luatex or xetex
  \usepackage{unicode-math}
  \defaultfontfeatures{Scale=MatchLowercase}
  \defaultfontfeatures[\rmfamily]{Ligatures=TeX,Scale=1}
\fi
\usepackage{lmodern}
\ifPDFTeX\else  
    % xetex/luatex font selection
\fi
% Use upquote if available, for straight quotes in verbatim environments
\IfFileExists{upquote.sty}{\usepackage{upquote}}{}
\IfFileExists{microtype.sty}{% use microtype if available
  \usepackage[]{microtype}
  \UseMicrotypeSet[protrusion]{basicmath} % disable protrusion for tt fonts
}{}
\makeatletter
\@ifundefined{KOMAClassName}{% if non-KOMA class
  \IfFileExists{parskip.sty}{%
    \usepackage{parskip}
  }{% else
    \setlength{\parindent}{0pt}
    \setlength{\parskip}{6pt plus 2pt minus 1pt}}
}{% if KOMA class
  \KOMAoptions{parskip=half}}
\makeatother
\usepackage{xcolor}
\setlength{\emergencystretch}{3em} % prevent overfull lines
\setcounter{secnumdepth}{5}
% Make \paragraph and \subparagraph free-standing
\makeatletter
\ifx\paragraph\undefined\else
  \let\oldparagraph\paragraph
  \renewcommand{\paragraph}{
    \@ifstar
      \xxxParagraphStar
      \xxxParagraphNoStar
  }
  \newcommand{\xxxParagraphStar}[1]{\oldparagraph*{#1}\mbox{}}
  \newcommand{\xxxParagraphNoStar}[1]{\oldparagraph{#1}\mbox{}}
\fi
\ifx\subparagraph\undefined\else
  \let\oldsubparagraph\subparagraph
  \renewcommand{\subparagraph}{
    \@ifstar
      \xxxSubParagraphStar
      \xxxSubParagraphNoStar
  }
  \newcommand{\xxxSubParagraphStar}[1]{\oldsubparagraph*{#1}\mbox{}}
  \newcommand{\xxxSubParagraphNoStar}[1]{\oldsubparagraph{#1}\mbox{}}
\fi
\makeatother


\providecommand{\tightlist}{%
  \setlength{\itemsep}{0pt}\setlength{\parskip}{0pt}}\usepackage{longtable,booktabs,array}
\usepackage{calc} % for calculating minipage widths
% Correct order of tables after \paragraph or \subparagraph
\usepackage{etoolbox}
\makeatletter
\patchcmd\longtable{\par}{\if@noskipsec\mbox{}\fi\par}{}{}
\makeatother
% Allow footnotes in longtable head/foot
\IfFileExists{footnotehyper.sty}{\usepackage{footnotehyper}}{\usepackage{footnote}}
\makesavenoteenv{longtable}
\usepackage{graphicx}
\makeatletter
\def\maxwidth{\ifdim\Gin@nat@width>\linewidth\linewidth\else\Gin@nat@width\fi}
\def\maxheight{\ifdim\Gin@nat@height>\textheight\textheight\else\Gin@nat@height\fi}
\makeatother
% Scale images if necessary, so that they will not overflow the page
% margins by default, and it is still possible to overwrite the defaults
% using explicit options in \includegraphics[width, height, ...]{}
\setkeys{Gin}{width=\maxwidth,height=\maxheight,keepaspectratio}
% Set default figure placement to htbp
\makeatletter
\def\fps@figure{htbp}
\makeatother
% definitions for citeproc citations
\NewDocumentCommand\citeproctext{}{}
\NewDocumentCommand\citeproc{mm}{%
  \begingroup\def\citeproctext{#2}\cite{#1}\endgroup}
\makeatletter
 % allow citations to break across lines
 \let\@cite@ofmt\@firstofone
 % avoid brackets around text for \cite:
 \def\@biblabel#1{}
 \def\@cite#1#2{{#1\if@tempswa , #2\fi}}
\makeatother
\newlength{\cslhangindent}
\setlength{\cslhangindent}{1.5em}
\newlength{\csllabelwidth}
\setlength{\csllabelwidth}{3em}
\newenvironment{CSLReferences}[2] % #1 hanging-indent, #2 entry-spacing
 {\begin{list}{}{%
  \setlength{\itemindent}{0pt}
  \setlength{\leftmargin}{0pt}
  \setlength{\parsep}{0pt}
  % turn on hanging indent if param 1 is 1
  \ifodd #1
   \setlength{\leftmargin}{\cslhangindent}
   \setlength{\itemindent}{-1\cslhangindent}
  \fi
  % set entry spacing
  \setlength{\itemsep}{#2\baselineskip}}}
 {\end{list}}
\usepackage{calc}
\newcommand{\CSLBlock}[1]{\hfill\break\parbox[t]{\linewidth}{\strut\ignorespaces#1\strut}}
\newcommand{\CSLLeftMargin}[1]{\parbox[t]{\csllabelwidth}{\strut#1\strut}}
\newcommand{\CSLRightInline}[1]{\parbox[t]{\linewidth - \csllabelwidth}{\strut#1\strut}}
\newcommand{\CSLIndent}[1]{\hspace{\cslhangindent}#1}

\KOMAoption{captions}{tableheading}
\makeatletter
\@ifpackageloaded{bookmark}{}{\usepackage{bookmark}}
\makeatother
\makeatletter
\@ifpackageloaded{caption}{}{\usepackage{caption}}
\AtBeginDocument{%
\ifdefined\contentsname
  \renewcommand*\contentsname{Table of contents}
\else
  \newcommand\contentsname{Table of contents}
\fi
\ifdefined\listfigurename
  \renewcommand*\listfigurename{List of Figures}
\else
  \newcommand\listfigurename{List of Figures}
\fi
\ifdefined\listtablename
  \renewcommand*\listtablename{List of Tables}
\else
  \newcommand\listtablename{List of Tables}
\fi
\ifdefined\figurename
  \renewcommand*\figurename{Figure}
\else
  \newcommand\figurename{Figure}
\fi
\ifdefined\tablename
  \renewcommand*\tablename{Table}
\else
  \newcommand\tablename{Table}
\fi
}
\@ifpackageloaded{float}{}{\usepackage{float}}
\floatstyle{ruled}
\@ifundefined{c@chapter}{\newfloat{codelisting}{h}{lop}}{\newfloat{codelisting}{h}{lop}[chapter]}
\floatname{codelisting}{Listing}
\newcommand*\listoflistings{\listof{codelisting}{List of Listings}}
\makeatother
\makeatletter
\makeatother
\makeatletter
\@ifpackageloaded{caption}{}{\usepackage{caption}}
\@ifpackageloaded{subcaption}{}{\usepackage{subcaption}}
\makeatother

\ifLuaTeX
  \usepackage{selnolig}  % disable illegal ligatures
\fi
\usepackage{bookmark}

\IfFileExists{xurl.sty}{\usepackage{xurl}}{} % add URL line breaks if available
\urlstyle{same} % disable monospaced font for URLs
\hypersetup{
  pdftitle={Hubbard Brook Information Management Documents},
  colorlinks=true,
  linkcolor={blue},
  filecolor={Maroon},
  citecolor={Blue},
  urlcolor={Blue},
  pdfcreator={LaTeX via pandoc}}


\title{Hubbard Brook Information Management Documents}
\author{}
\date{}

\begin{document}
\maketitle

\renewcommand*\contentsname{Table of contents}
{
\hypersetup{linkcolor=}
\setcounter{tocdepth}{2}
\tableofcontents
}

\bookmarksetup{startatroot}

\chapter*{Preface}\label{preface}
\addcontentsline{toc}{chapter}{Preface}

\markboth{Preface}{Preface}

This collection of documentation contains details on Information
Management for the Hubbard Brook Experimental Forest. Data are managed
by both the USDA Forest Service (USFS) and the Information Manager for
the Hubbard Brook LTER site (one of 25+ sites funded by NSF's Long-term
Ecological Research program).

In addition to an overview of information management at HBR, we include
supplemental chapters and associated documents covering more dynamic
details such as inventory/status of data packages, current IM projects
and timelines, a guide to the operation of the HBR website, and a
step-by-step guide to the development of HBR data packages and
associated local data catalog. This document is revised periodically to
reflect changes in IM assets, status, workflows, etc.

Throughout these pages you will encounter a number of acronyms. They are
typically identified before first use within a chapter\ldots but just in
case:

\begin{itemize}
\tightlist
\item
  EDI: Environmental Data Initiative\\
\item
  EML: Ecological Metadata Language\\
\item
  ESRC: Earth Systems Research Center\\
\item
  HBEF: Hubbard Brook Experimental Forest\\
\item
  HBR: Acronym for the Hubbard Brook LTER\\
\item
  LTER: Long-term Ecological Research\\
\item
  LNO: LTER Network Office\\
\item
  PASTA+:~The software that runs the repository
\item
  RAC: Research Advisory Committee\\
\item
  RCC: Research Computing Center\\
\item
  USFS: USDA Forest Service\\
\item
  UNH: University of New Hampshire\\
\item
  WMNF: White Mountain National Forest\\
\end{itemize}

\bookmarksetup{startatroot}

\chapter{Data Management Plan}\label{data-management-plan}

This chapter is based on the most recent Data Management Plan (DMP)
developed for the 2022 LTER site renewal proposal. Subsequent chapters
expand on each element and reflect recent accomplishments and detailed
instructions on the various workflows used in the HBR Information
Management environment. Also available is the full
\href{2022_HBRSupplementalDoc_DMP-20220317.pdf}{2022 HBR DMP} as
originally submitted.

\section{History of IM at the Hubbard Brook Experimental
Forest}\label{history-of-im-at-the-hubbard-brook-experimental-forest}

The Hubbard Brook Experimental Forest (HBEF; USDA Forest Service) was
established in 1955 and became an NSF-funded Long Term Ecological
Research Site (HBR) in 1988. Information management has been an
important component at Hubbard Brook from its inception. Data and
documents from 1955 onward have been stored and protected, and although
most of these early items consist of physical assets (paper charts,
photographic slides, field notes, handwritten data, publications, etc),
much of this material has been converted to digital format, with
original copies in fireproof storage at the HBEF Pierce Lab and at the
Northern Research Station in Durham, NH.

The establishment of the LTER-HBR occurred at a time of rapidly changing
technology; desktop computers and email were new, and the internet as we
know it was still several years away. The Hubbard Brook community fully
embraced these emerging technological resources, and established access
to data with the `Source of the Brook', a public access dial-up
electronic bulletin board, which allowed easy retrieval of many data
sets from the HBR (1990). From a dialup bulletin board and gopher server
in the early 1990s, to the World Wide Web in the late 1990s, HBR's
latest technology advances in publicly sharing data and resources has
seen a migration of the website to WordPress, a data catalog built on
dynamic access to content in the Environmental Data Initiative
repository (EDI), bibliography management in Zotero, to name a
few\ldots{}

Until 2012, Information management for HBR was provided through the
Forest Service, with John Campbell filling this role from 1997-2012.
During this time, the LTER network adopted EML (Ecological Metadata
Language) as a metadata standard, and HBR was an early adopter of this
standard. In 2003-4 the first EML-based data packages were prepared for
HBR with online download access and formatted browser display of
metadata.

Funding for the HBR Information Management position was provided in the
2010 renewal of LTER-HBR funding (HBR5), and the position was filled in
mid-2012 by Mary Martin (Earth Systems Research Center, University of
New Hampshire, Durham, NH).

\section{Governance}\label{governance}

Information management at the Hubbard Brook Ecosystem Study (HBES) is
guided by the Information Oversight Committee (IOC), which meets on an
ad hoc basis with virtual IOC meetings scheduled accordingly.

\section{Research Approval Committee}\label{research-approval-committee}

A Research Approval Committee (RAC) has been established to assist the
Forest Service and broader HBES community in making decisions regarding
what research studies will be allowed. In making its recommendation, the
RAC considers a number of factors related to: (1) the relationship of
the proposed project to the overall Hubbard Brook Ecosystem Study (how
does this project fit into the overall study; why is it important for
this research to occur at the Hubbard Brook Experimental Forest, as
opposed to some other site); (2) the scientific merit of the proposed
research; (3) the integrity of the site (e.g.~how will this research
impact the Forest or other ongoing research projects); and (4) the
extent to which the proposed research compromises or enhances ongoing
efforts. The RAC's critical review of proposed research at Hubbard Brook
helps ensure that the scientific value of the Hubbard Brook Experimental
Forest is maintained for the future.

Proposal submissions to the RAC are made through JotForm webforms on
\url{https://hubbardbrook.org/research/research-proposal-submission},
and are currently being used to develop a project-level database to
support the RAC review process and IM tracking of data collections at
Hubbard Brook. Researchers approved by the RAC are encouraged to submit
data to be included in the Hubbard Brook Data Catalog, regardless of
funding source.

\bookmarksetup{startatroot}

\chapter{Personnel}\label{personnel}

Hubbard Brook LTER Information management transitioned from the Forest
Service to the HBR LTER grant in the HBR-V funding cycle (2010-2016).
The Information Manager is based at the Earth System Research Center
(ESRC), University of New Hampshire (UNH), and funded through a
subcontract between the Cary Institute of Ecosystem Studies and UNH.

Information Management resources (software, hardware, and personnel)
from the USDA Forest Service Northern Research station contribute
substantially to the overall data holdings of the HBR. These include the
collection, quality control, and development of core hydrological,
meteorological, and water chemistry datasets upon which much of the HBR
research relies.

See also IT resources for as-needed project support (Section 4) and
Appendix 1 for a description of as-needed support from the ESRC
\emph{Laboratory for Remote Sensing and Spatial Analysis.}

\bookmarksetup{startatroot}

\chapter{Data Packages}\label{data-packages}

\section{Overview}\label{overview}

HBR data packages are prepared for submission to the Environmental Data
Initiative (EDI), following best practices developed over 4 decades by
the LTER Information Management community
(\url{https://ediorg.github.io/data-package-best-practices/eml-best-practices.html}).
These best practices, and the efforts of the EDI, ensure that data are
Findable, Accessible, Interoperable, and Reusable, following the
principles of the
\href{https://www.go-fair.org/go-fair-initiative}{FAIR} initiative. The
EDI serves as the primary repository for HBR data, and details about the
operation of EDI can be found at
\href{https://edirepository.org/}{https://edirepository.org}.

\section{Data Holdings}\label{data-holdings}

The HBR data catalog has had a strong emphasis on long-term datasets and
data from the major watershed experiments. Many of these data pre-date
the establishment of LTER-HBR, and are now available as a result of a
60+ year culture of robust data curation and sharing. More than 20 HBR
data packages have been collected over a period of 50 or more years,
with another 30 covering a timespan of more than 20 years. Through close
coordination with the Research Approval Committee, the Hubbard Brook
Committee of Scientists, and project administration, HBR-IM is able to
identify datasets that can be incorporated into our data catalog.
Graduate students working at Hubbard Brook are also surveyed
periodically to identify forthcoming datasets and they are also trained
in the EDI data publication workflow.

\section{Metadata Standards}\label{metadata-standards}

All HBR data packages are prepared for submission through the
development of metadata in the Ecological Metadata Language standard
(EML;
\url{https://eml.ecoinformatics.org/eml-ecological-metadata-language}).
Basic EML content includes: title, abstract, personnel, contacts,
publication date, spatial and temporal coverages, keywords (consistent
with LTER controlled vocabulary), project funding, publisher, data
access and use policies, and detailed attribute-level metadata. Data
download and use is facilitated through the fully described attribute
metadata (column names, definitions, units, missing values, and coding).
The highest level of EML completion is achieved through the EDI
congruency checker with informational, warning, and error messages that
provide feedback to HBR-IM on additional steps that can (or must) be
taken to submit to the repository. These congruency checks read both
metadata and data, testing a minimum of 40 conditions that can be
addressed to insure that data packages are are fully capable of
integration with other data, and fully operational in higher level
workflows and automated data processing.

In 2019, LTER sites began using EML2.2. EML2.2 provides the structure to
accommodate a number of advanced metadata elements. Of note, is the
ability to annotate data packages at the data package, entity, and
attribute level, by linking to persistent identifiers in ontologies,
such as those found at
\url{https://bioportal.bioontology.org/ontologies} and elsewhere. As the
annotation elements in EML2.2 become populated, both the discoverability
of datasets, and the ability to use datasets in synthesis efforts will
be enhanced. HBR-IM has been a member of both the EDI Semantics Working
Group (formed in January 2019), and the EDI/LTER Units Working group
(2023-present), which have a goal of developing best practices and
training on the use of the new EML2.2 annotation elements. The Units
Working Group is responsive to recommendations from the LTER 40-year
review on facilitating synthesis efforts. This has been accomplished by
establishing a relationship with the QUDT ontology
(\url{https://qudt.org),} developing code to map ad hoc units to this
ontology, and preparing a manuscript on this effort for publication.

\section{Data Package Development}\label{data-package-development}

HBR IM has fully adopted the EDI ezEML application for data package
development. This cloud-based application was developed by EDI and users
are supported by the EDI team and ezEML developer. This application
continues to be developed to support emerging LTER/EDI data
requirements. Within this system, HBR can store templates for access and
re-use of common metadata elements (people, taxonomy, geographic
coverage, funding, etc). EzEML also supports a `collaboration' mode,
where IM and data creators can work together to complete the full
metadata documentation necessary for the repository.

A separate chapter describes the data package development workflow in
detail: \href{DataPackageWorkflow.qmd}{HBR Data Package Development}.

\section{Data Quality Control}\label{data-quality-control}

The HBR research community is widely dispersed among different
institutions and laboratories, and data quality control is implemented
primarily by the individual researcher. All data packages include
methods, wherein detailed data QC protocols can be documented. The
HBR-IM works with research teams to document quality control in the data
package metadata as appropriate. This may range from descriptions
directly in EML, PDF files uploaded with data packages, or cross
referencing to details on data QC available elsewhere. IM provides the
data submitter with feedback on a number of QC checks that are
implemented during the data package development workflow. These include
value ranges for data table attributes, coding consistency, and
additional issues that are flagged within the ezEML environment and the
EDI congruency checker (the final checks during repository upload).

\section{Access Policy}\label{access-policy}

The HBR data policy follows that of the LTER Data Access Policy, as
updated in 2017. (\url{https://lternet.edu/data-access-policy}/;
Creative Commons license - Attribution - CC BY;
\url{https://creativecommons.org/licenses/by/4.0/}). All pre-existing
HBR data packages have been revised to include this new policy, and the
policy is linked on the hubbardbrook.org information management page.
The policy reads as follows:

\begin{quote}
\emph{This information is released under the Creative Commons license -
Attribution - CC BY
(\url{https://creativecommons.org/licenses/by/4.0/}). The consumer of
these data (``Data User'' herein) is required to cite it appropriately
in any publication that results from its use. The Data User should
realize that these data may be actively used by others for ongoing
research and that coordination may be necessary to prevent duplicate
publication. The Data User is urged to contact the authors of these data
if any questions about methodology or results occur. Where appropriate,
the Data User is encouraged to consider collaboration or co-authorship
with the authors. The Data User should realize that misinterpretation of
data may occur if used out of context of the original study.}

\emph{While substantial efforts are made to ensure the accuracy of data
and associated documentation, complete accuracy of data sets cannot be
guaranteed. All data are made available ``as is.'' The Data User should
be aware, however, that data are updated periodically and it is the
responsibility of the Data User to check for new versions of the data.
The data authors and the repository where these data were obtained shall
not be liable for damages resulting from any use or misinterpretation of
the data.}
\end{quote}

\section{Data Access}\label{data-access}

The complete inventory of Hubbard Brook data can be browsed, filtered,
and searched on the HBR website
\url{https://hubbardbrook.org/d/hubbard-brook-data-catalog}. Data are
also discoverable through both the EDI data portal
(\href{https://portal.edirepository.org/}{https://portal.edirepository.org}),
through DataONE
(\href{https://search.dataone.org/}{https://search.dataone.org}), google
dataset search (\url{https://datasetsearch.research.google.com/}), and
through DataCite (\href{https://datacite.org/}{https://datacite.org}),
the entity providing dataset DOIs to EDI. A separate stand-alone
document is generated as needed (for proposals and reviews), and
describes all HBR data packages in the EDI (and other) repository --
\textbf{\emph{Hubbard Brook Data Catalog Inventory}}.

Also see ESRC Computer Resources appendix 1.

Table 1 outlines software in use by HBR-IM to manage data package
development and the Hubbard Brook website.

Table 1. Features of HBR Information Management System

\begin{longtable}[]{@{}
  >{\raggedright\arraybackslash}p{(\columnwidth - 2\tabcolsep) * \real{0.5139}}
  >{\raggedright\arraybackslash}p{(\columnwidth - 2\tabcolsep) * \real{0.4861}}@{}}
\toprule\noalign{}
\begin{minipage}[b]{\linewidth}\raggedright
\textbf{Feature}
\end{minipage} & \begin{minipage}[b]{\linewidth}\raggedright
\textbf{Details, software, resources}
\end{minipage} \\
\midrule\noalign{}
\endhead
\bottomrule\noalign{}
\endlastfoot
Website: \href{https://hubbardbrook.org/}{https://hubbardbrook.org} &
WordPress, html, css, php, xslt, javascript, apache, piwigo \\
Bibliography & Zotero, Zotpress wordpress plugin \\
Data Catalog & EDI data repository, local WordPress gateway to EDI HBR
data, EDIutils R \\
Metadata & ezEML, PostgreSQL, EML R package, EML2.1 \\
Computer Hardware & Dell Poweredge R510, desktop and laptop linux
systems. \\
Backup & BackupPC, rsnapshot, daily, weekly and monthly backups, on and
off-site \\
Data management & R, LibreOffice, QGIS, MySQL, PostgreSQL, git \\
\end{longtable}

\bookmarksetup{startatroot}

\chapter{Website}\label{website}

The HBR website
(\href{https://hubbardbrook.org/}{https://hubbardbrook.org}) is the
primary means by which HBR information is disseminated, with additional
non-digital data (charts, maps, photographs) made available upon
request. HBR completed a website migration from html/php files to a
content management system (CMS; Drupal) in 2017. In 2022 the website was
further migrated to WordPress. With cloud-based WordPress hosting, it
becomes simpler to transfer website management to a different IM and/or
institution. Website content is managed by the HBR IM and the Hubbard
Brook Research Foundation. The website provides access to data,
publications, personnel pages, education and outreach material, and a
photo gallery. A recent website content addition is the Hubbard Brook
Research Synthesis -- an online `book' consisting of 19 chapters to
date, covering a wide range of long-term research. With content editors
assigned to each chapter, this is meant to be a series of dynamic pages,
which reflect the full history of Hubbard Brook research in each topic
area (\url{https://hubbardbrook.org/online-book}). Chapters in the
online book contain numerous data figures, many of which have been
restructured to read data directly from the most recent data revisions
in the EDI repository.

See \textbf{\emph{Hubbard Brook Ecosystem Study Website Management
Guide}} for website configuration, access to servers and filesystems,
recommended practices for site content, etc.

\bookmarksetup{startatroot}

\chapter{Sample Archive}\label{sample-archive}

\section{Samples}\label{samples}

In 1990, an archive facility was built at the Hubbard Brook Experimental
Forest to store samples permanently so that they will be available for
future research. The 1860 sq. ft. building consists of two rooms: a
larger unheated room (30 x 46 ft.) and a smaller room (16 x 30 ft.)
heated to just above freezing in the winter. The larger room is
uninsulated and is subject to large variations in temperature and
humidity; the most scientifically valuable samples are stored in the
smaller, insulated, heated room.

The archive building now houses approximately 70,000 samples of soil,
water, plant tissue, and other materials. Samples are preserved,
barcoded, and cataloged with accompanying metadata in a database. This
database of bar-coded samples is searchable online at
\url{https://data.hubbardbrook.org/samples/}. In 2024, both the
underlying collection and sample database and the search interface are
being restructured. This effort is resulting in improved collection
organization and more detailed sample-level metadata.

Requests for reanalysis of archived samples (e.g.~isotopic analyses,
heavy metals, etc.) are received periodically, and have resulted in a
number of publications.

A Sample Archive Committee (SAC) was formed in 2013 to address storage
space, future direction, priority for continued bar-coding, etc.

\section{Sample Archive Subsampling
Policy}\label{sample-archive-subsampling-policy}

HBR shares these archived samples with scientists upon request. As
stewards of these samples, our highest priorities are:

\begin{enumerate}
\def\labelenumi{\arabic{enumi}.}
\item
  to maintain the chemical integrity of these samples;
\item
  to document the use of these samples, and any resulting changes;
\item
  inform principal investigators of interest in using them;
\item
  to acknowledge the appropriate funding sources for their original
  collection.
\end{enumerate}

Details on the subsampling of archived material can be found on the
sample request form:

\url{https://hubbardbrook.org/sites/default/files/documents/subsampling_request.pdf}

\section{Directions for Sample
Submission}\label{directions-for-sample-submission}

Requirements for acceptance of samples into the archive:

\begin{enumerate}
\def\labelenumi{\arabic{enumi}.}
\item
  Adequate documentation must accompany physical samples.
\item
  Samples are stored in either an unheated large room or a smaller room
  that is heated to just above freezing. The contributing scientist is
  responsible for deciding that these conditions are suitable for
  his/her samples.
\item
  Soil samples must be air or oven-dried and stored in plastic or glass
  bottles with screw caps to ensure a tight seal. Cardboard is not
  permitted.
\item
  Vegetation samples should be dried, ground and stored in clear plastic
  or glass containers.
\item
  Water samples must be stored in plastic bottles and will be accepted
  either treated, or not. If treated, the investigator must specify the
  type and concentration used.
\item
  All tree logs, cookies and cores should be air-dried and can be stored
  in cardboard boxes or arranged in a manner that will allow for air to
  flow between individual samples. Tree cores should be mounted or
  stored in straws.
\item
  Samples that are considered toxic may be rejected. The data management
  committee may confer with the SAC about important, but toxic samples
  requiring storage.
\end{enumerate}

\bookmarksetup{startatroot}

\chapter{Bibliography}\label{bibliography}

The current bibliography for Hubbard Brook includes Books, Journal
Articles, Conference Presentations, and Theses; currently with more than
2400 entries.

Citations are managed locally in Zotero
(\href{https://zotero.org/}{https://zotero.org}). This open source
bibliography management software allows for export to standard reference
management exchange formats, harvests citation information easily
through a browser, and provides cite-as-you-write support for MSWord and
Open/LibreOffice.

The Zotero bibliography is mirrored to the cloud, and publications are
accessed through a link to this service. The WordPress zotpress plugin
provides the functionality of linking each HB researcher's profile page
to their Hubbard Brook publications.

The Hubbard Brook bibliography is also mirrored to the LTER Network
Communication Office (NCO) central LTER bibliographic database.

\bookmarksetup{startatroot}

\chapter{Appendix 1. ESRC/UNH Computing
Facilities}\label{appendix-1.-esrcunh-computing-facilities}

This section has been expanded and included now as a
\href{UNH_computer_support.qmd}{separate chapter}.

\bookmarksetup{startatroot}

\chapter{Data Package Workflow}\label{data-package-workflow}

\section{Overview}\label{overview-1}

The purpose of this document is to capture details of the data package
development workflow that is currently in use at Hubbard Brook. HBR data
is published in the \href{https://edirepository.org}{Environmental Data
Initiative Repository (EDI)}. With the availability of EDI's ezEML data
package builder application (adopted by HBR in 2024), this once long and
complicated process has been greatly simplified.

In 2024, HBR fully adopted the EDI ezEML workflow for data package
development. All data packages are developed under the EDI HBR user
account. The division of effort varies with the nature of the data
package. Graduate students are encouraged to collaborate with the IM
online within the ezEML environment. This serves as a way to directly
input metadata without first populating a template, reduces IM time on
some data packages, and is an important skill-builder for HBR graduate
students.

EzEML provides the capability to store often-used metadata components in
a template. For HBR, there is one master template and additional
templates for HB projects that are frequent publishers (MELNHE, HBWaTER,
BIRD, CRCH). Since the master template is very extensive, it is not
cloned as a starting point for a new dataset (which might be a common
template use), but instead accessed through the import {[}creator,
geographic, keywords, project, funding{]} buttons where just selected
items are brought into the current dataset.

\section{Data package development
workspace}\label{data-package-development-workspace}

The working directory for package development is on the HBR-IM desktop
with the home directory for data package managment identified elsewhere
as \$DPM\_HOME.

Assets for each data package are in folders named
\$DPM\_HOME/ezEML/hbr{[}pkgid{]}. While most of the workflow occurs in
the ezEML environment, this local filesytem is used to handle dataset
assets submitted to IM (metadata templates, datafiles, etc). The
completed ezEML packages are downloaded (as zip) to this location for
subsequent upload to the EDI staging and production servers.

\section{Step-by-Step Data Package
Workflow}\label{step-by-step-data-package-workflow}

The HBR Data Inventory table is hosted on the HubbardBrook sharepoint
site (HBR-IM administrator at UNH). This table contains additional
information that we use on our local data catalog to enhance user
experience (flagging of significant core datasets, more robust LTER Core
Research Area assignments that may be missing in older metadata, and a
code to categorize datasets and to sort them in the initial catalog
view). Data packages are entered in this table as soon as they are
identified (in some cases with very long lead times). Upon becoming
aware of a dataset, a package id is assigned and the entry initiated
with status=anticipated. As soon as data and/or metadata are in-hand,
the status is updated to `draft'. The table includes packageID,
abbreviated title, contact, notes as needed, flagging as long-term core
dataset, and EDI submission status.

The steps are as follows:

\begin{itemize}
\tightlist
\item
  Components for data packages are provided to the IM through a
  sharepoint dropoff.
\item
  ezEML collaboration is established if desired for the dataset.
\item
  A data package is initiated in ezEML with the naming convention of
  hbr{[}pkgid{]}-{[}shortname{]}.
\item
  The HBR Master Template (stored in EDI) is used to import people,
  geographic areas, keywords, projects, and funding.
\item
  Data tables are loaded from csv.
\item
  Data table attributes are documented either directly on the online
  forms or through the ezEML table entity templating feature (a great
  timesaver for complex datatables).
\item
  ezEML data package is downloaded to IM's computer
\item
  The R script to add QUDT unit annotations is run
\item
  The annotated eml file is uploaded to portal-s.lternet.edu and the URL
  is shared with creator for review
\item
  Subsequent edits are made with creator feedback
\item
  Data package is approved by the creator
\item
  Data package is uploaded to the live repository
\end{itemize}

\section{Non-tabular datasets (images, audio, very large datasets,
etc)}\label{non-tabular-datasets-images-audio-very-large-datasets-etc}

Hubbard Brook has published a number of datasets that contain zip files
of pdfs, images, audio files, etc. Guidance for preparing these special
case datasets can be found in the
\href{https://ediorg.github.io/data-package-best-practices/data-package-design-for-special-cases.html}{EML
Best Practices document}.

\subsection{Large Datasets}\label{large-datasets}

In some of these cases, the data entities are quite large and cannot be
uploaded with the browser interface (500Mb max), but are within the size
cap for online EDI data storage. At the current time, large datasets are
staged on a UNH server with the distribution URL set to that location.
In some cases we develop packages using a smaller placeholder file so
that we do not overload storage on the ezEML platform or portal-s
staging area. Datasets exceeding the 100Gb threshold are deemed ``too
large for HTTP'' and must be prepared as offline data entities.

\section{Notes on revising older
datasets.}\label{notes-on-revising-older-datasets.}

When earlier data packages are revised, the starting point is an ezEML
fetch of the published data package. Steps are similar to those used for
a new dataset, but remember to increment the revision number. Assets for
earlier data packages developed can be found in either the
`EMLassemblyline' or `legacy' folders, although those files should
rarely be necessary once a data package is published in EDI.

An EDI fetched dataset may have been developed with a non-ezEML
workflow. If that is the case, items requiring attention will be:

\begin{itemize}
\tightlist
\item
  Creators -- delete and import from the template to be sure to get
  ORCIDs and institution RORs for each person. Use the `sort' function
  on people import to find them easier in the long list.
\item
  Project -- EMLAL datasets may have funding in `related funding' or a
  text string in project abstract. Delete these. Import all funding from
  the template as primary or related. The template will have enhanced
  information to include grant url, funding agency ROR, etc.
\item
  Intellectual rights will be correct for all older datasets, but it is
  best to reset that in ezEML to CC-BY selection.
\item
  Increment the packageId revision number.
\item
  Use re-upload datatable if revision includes new or modified data.
  This all goes well if the table is identical. If there are new
  columns, upload as a new table and clone metadata from the original,
  then define any new columns. If the dataset was prepared in EMLAL,
  clear min/max bounds.
\end{itemize}

\bookmarksetup{startatroot}

\chapter{Forest Service Data
Workflow}\label{forest-service-data-workflow}

The staff at the Hubbard Brook Experimental Forest manage the entire
data lifecycle for many of the long-term datasets, These include
hydrology, meteorology, phenology, and others. These data are prepared
for the EDI repository using EDI's EML Assemblyline workflow.

\bookmarksetup{startatroot}

\chapter{Data Catalog Workflow}\label{data-catalog-workflow}

\section{Overview}\label{overview-2}

The purpose of this document is to capture details of the workflow to
build a data catalog table that is used on the
\href{https://hubbardbrook.org/data-catalog}{Hubbard Brook Data Catalog}
displayed on our website. This workflow has been reconfigured from
earlier versions to now read only from publicly available sources - EDI
and a public sharepoint file with enhanced data package details. The
latter improve the user experience for data searchers by categorizing
HBR data and adding robust LTER core area tags.

\section{Database and File access}\label{database-and-file-access}

Access to dataset details is provided by the LTER PASTA API via the
EDIutils R package. The enhanced table resides on the HubbardBrook
sharepoint site (HBR IM admin, UNH). The sharepoint data inventory file
is maintained to track status of each dataset and to provide additional
information that is not included in the formal metadata or may be
lacking in very old datasets but is useful in our data catalog (LTER
Core Research Area, HBR significant data status)

\section{Step-by-Step Catalog
Workflow}\label{step-by-step-catalog-workflow}

\begin{itemize}
\tightlist
\item
  Run the code in Appendix A (dataCat.R)
\item
  log in to the wordpress site
\item
  open the data catalog table
\item
  upload wptablefeed.csv to replace existing version
\end{itemize}

\section{APPENDIX A -- Code to build the local HBR data
catalog:}\label{appendix-a-code-to-build-the-local-hbr-data-catalog}

Will be setting this up in github to replace this snapshot here

\section{FetchSarepointDataInventory.R}\label{fetchsarepointdatainventory.r}

\begin{verbatim}
##########################################
# FetchSharepointDataInventory.R
########################################## 

library(httr)
library(readxl)

# Function to read Excel file from SharePoint URL
read_sharepoint_excel \<- function(url) {

# Download the file
response \<- GET(url, config(ssl_verifypeer = FALSE))

# Check if the download was successful
if (status_code(response) != 200) {
  stop("Failed to download the file. Status code: ", status_code(response))
}

# Create a temporary file
temp_file \<- tempfile(fileext = ".xlsx")

# Write the content to the temporary file
writeBin(content(response, "raw"), temp_file)

# Read the Excel file
df \<- read_excel(temp_file)

# Remove the temporary file
unlink(temp_file)

return(df)

}

# URL of your SharePoint Excel file
url \<- "[https://universitysystemnh.sharepoint.com/:x:/t/HubbardBrook/EXFAJdG37VxOsrGI5jMhxmcBc1vTpUnZw1WPtiP3Q5Mc4A?download=1"](https://universitysystemnh.sharepoint.com/:x:/t/HubbardBrook/EXFAJdG37VxOsrGI5jMhxmcBc1vTpUnZw1WPtiP3Q5Mc4A?download=1")

# Read the Excel file
df \<- data.frame(read_sharepoint_excel(url))

# subset to just cataloged datasets, save as df 'ps' (for package
state)

ps=df\[which(df\$status=="cataloged"),\]

}
\end{verbatim}

\section{dataCat.R}\label{datacat.r}

\begin{verbatim}
#############################################
# dataCat.R
#
# 20241016
# Purpose: build a datatable for the HB website local data catalog
# This is a revised script to generate the data catalog file on the
wordpress site
# this revision now runs on all publicly available data - no pw
protected databases for local info
# Inputs are - sharepoint xlsx file that tracks data status locally and
provides additional info to enhance user
# experience in searching for data

# Requirements: The following script is sourced and should be located
in the same folder as this main script: FetchSharepointDataInventory.R

# Usage: Run this script, then move the wptablefeed.csv file to the
website and updated the table to refresh to this new version

# Note: once this runs a few times and I gain a little confidence, I
will consider adding a command to ftp directly to wordpress to refresh
the site
#############################################

library(EDIutils)
library(tidyverse)
library(httr)
library(\"stringr\")
library(xml2)

# set the working directory to the location of the script
setwd(dirname(rstudioapi::getActiveDocumentContext()\$path))

# source the script that gets local package info from sharepoint
spreadsheet

# returns dataframe ps(package state)
source(\"FetchSharepointDataInventory.R\")

# Fetch the basic eml info directly from EDI for each package
# consider getting abstract and making that display as a hoverover on
the website table

res\<-search_data_packages(

query =
\"q=\*&fq=-scope:ecotrends&fq=scope:knb-lter-hbr&fq=-scope:lter-landsat\*&fl=id,packageid,doi,title,pubdate,begindate,enddate\")

names(res)=c(\"id\",\"PackageId\",\"doi\",\"Title\",\"pubdate\",\"begindate\",\"enddate\")

res\$Title=gsub(\"\[\\r\\n\]\", \" \", res\$Title)

res\$Title=gsub(\"\\\\s+\", \" \", res\$Title)

# extract begin and end YEAR
res\$startYear=as.POSIXlt(as.Date(res\$begindate))\$year + 1900
res\$endYear=as.POSIXlt(as.Date(res\$enddate))\$year + 1900

# if dataset has start/end dates, create column that shows them with
dash separator
res\$yearrange=\"NA\"
res\$yearrange = paste0(res\$startYear,\" - \",res\$endYear)

index= grep(\"NA\",res\$yearrange)
res\$yearrange\[index\] = \" \"

#create the edilink
res\$edilink=paste0(\"https://portal.edirepository.org/nis/mapbrowse?packageid=\",res\$PackageId)

# Fetch the pesky keywords and authors as xml so you can insert a
separator

k\<-search_data_packages(

query =
\"q=\*&fq=-scope:ecotrends&fq=scope:knb-lter-hbr&fq=-scope:lter-landsat\*&fl=id,packageid,keyword,author\",

as = \"xml\"

)

kw \<- data.frame(
PackageId=character(),
Originators=character(),
Keywords=character(),
stringsAsFactors=FALSE)

count=1

for (doc in xml_find_all(k, \".//document\")) {

print(doc)

# Get the keywords from the current doc
keyword_elements \<- xml_find_all(doc, \".//keyword\")
keyword_strings \<- xml_text(keyword_elements)

# This doesn\'t get the authors where HBWaTER and USFS are institution
authors.
# see solution below to get those from cite.edirepository.org

kw\[count,3\] \<- paste(keyword_strings, collapse = \";\")

# Get the authors from the current doc
author_elements \<- xml_find_all(doc, \".//author\")
author_strings \<- xml_text(author_elements)

kw\[count,2\] \<- paste(author_strings, collapse = \"; \")

\# get the packageid
pid \<- xml_find_all(doc, \".//packageid\")
pidstring \<- xml_text(pid)
kw\[count,1\] \<- pidstring
count=count+1

}

# tidy up keywords where some strings have newlines or consec white
spaces
kw\$Keywords=gsub(\"\[\\r\\n\]\", \" \", kw\$Keywords)
kw\$Keywords=gsub(\"\\\\s+\", \" \", kw\$Keywords)

# merge the tidier keywords with the main table
resj=merge(res,kw, by = \"PackageId\")

# get the dataset citations so that you have a nicer listing of authors
# you could probably do that where I do the keywords now from xml, but
it doesn\'t

# populate the records where author is an institution (USFS and
HBWaTER)

for(i in 1:dim(resj)\[1\]){

print(i)

CMD=paste0(\'GET
(\"https://cite.edirepository.org/cite/\',resj\[i,1\],\'\")\')

# sleep can be removed if you are whitelisted to make rapid EDI queries
Sys.sleep(0.5)

print(CMD)

c=content(eval(parse( text = CMD )))

print(c\$authors)

resj\[i,\"Originators\"\]=c\$authors

}

# FetchSharepointDataInventory returns ps dataframe (aka package state
in the originalmetabase database)

# add data category based on sort order codes in ps (local package
state table)

# apply category name to sort order values

ps\$category=0
index=which(substr(ps\$pub_notes,1,1)==1)
ps\$category\[index\]=\"Hydrometeorology\"
index=which(substr(ps\$pub_notes,1,1)==2)
ps\$category\[index\]=\"Water Chemistry\"
index=which(substr(ps\$pub_notes,1,1)==3)
ps\$category\[index\]=\"Soils\"
index=which(substr(ps\$pub_notes,1,1)==4)
ps\$category\[index\]=\"Vegetation\"
index=which(substr(ps\$pub_notes,1,1)==5)
ps\$category\[index\]=\"Heterotrophs\"
index=which(substr(ps\$pub_notes,1,1)==8)
ps\$category\[index\]=\"Documentation\"
index=which(substr(ps\$pub_notes,1,1)==9)
ps\$category\[index\]=\"Spatial Datasets\"

# subset out the columns that are to be used in the datatable
pscat=ps\[,c(\"dataset_archive_id\",\"category\",\"ltercore\",\"pub_notes\")\]

# merge the EDI query dataframe with ps
m=merge(resj,pscat, by.x=\"id\",by.y=\"dataset_archive_id\")

##### Write out wordpress data table \#################

# pull out the columns needed for website table
wptablefeed=m\[,c(\"PackageId\",\"Title\",\"Originators\",\"yearrange\",\"ltercore\",\"edilink\",\"category\",\"pub_notes\",\"Keywords\")\]

# sort packages based on pub_notes
wptablefeed.order=wptablefeed\[order(wptablefeed\$pub_notes),\]

# write out the table
write.table(wptablefeed.order,\"wptablefeed.csv\",row.names=FALSE,sep=\",\",na=\"
\")
\end{verbatim}

\bookmarksetup{startatroot}

\chapter{Data Inventory}\label{data-inventory}

A spreadsheet listing a complete inventory of HBR data, including
datasets anticipated and in draft format, is maintained on the UNH
HubbardBrook sharepoint site. Access to this sheet is shared by request.
This is used to build the local data catalog by merging with content
from the LTER PASTA API. It is also used by IM as a local record of what
is published, as well as those datasets anticipated, in draft format, or
staged for review. The emlWorkflow is used to identify datasets
developed by earlier workflows that could be updated to include eml
elements not available at time of publication (improved funding
metadata, general annotations, qudt unit annotations). The checkbox
columns for HBR\_V were used to modify our data table listing (proposal
supplemental document), by highlighting data used in papers during that
funding cycle and data used in the top10 publications that we
highlighted in the proposal.

Shared access to this data inventory table is view-only. Edits to the
table are limited to the site IM team. In view-only mode, some useful
search/filter includes filtering by project ID (used in the `nickname'),
by dataset status, core research area, etc. This provides a
comprehensive look at our data holdings to include datasets currently
under development as well as those anticipated on a longer timeline.

\begin{itemize}
\tightlist
\item
  DataSetID: just the pkg number
\item
  dataset\_archive\_id: packageId (knb-lter-hbr.XXX)
\item
  rev: revision number
\item
  nickname: a short name for the dataset\\
\item
  status: dropdown choice of anticipated, draft, staged, cataloged,
  deprecated, embargoed
\item
  emlWorkflow: legacy, EMLAL, ezEML, mmb (minimetabase)
\item
  notes
\item
  pub\_notes
\item
  dbupdatetime:carryover from mmb\\
\item
  update\_date\_catalog: date of last published revision\\
\item
  who2bug: contacts
\item
  in pasta: binary 0/1 to indicate published\\
\item
  signature: do we consider this a core HB longterm dataset
\item
  ltercore: core research areas (DP(disturbance process),IM(inorganic
  matter),OM(organic matter),PP(primary production),PS(population
  study)\\
\item
  hbr\_vcited: data cited in HBR-V publications\\
\item
  hbr\_vtop10: data cited in HBR-V top 10 publications
\end{itemize}

\bookmarksetup{startatroot}

\chapter{Publication Management}\label{publication-management}

\section{Overview}\label{overview-3}

Hubbard Brook publications are managed in Zotero.

\section{Zotero}\label{zotero}

Local IM desktop, tagging of publications, syncing to zotero cloud

\section{EDI publication-data
linkages}\label{edi-publication-data-linkages}

How this happens, reports we can generate from this

\section{Zotpress}\label{zotpress}

to display pubs on the people profile pages on website

\bookmarksetup{startatroot}

\chapter{Sample Archive Database}\label{sample-archive-database}

\section{Overview}\label{overview-4}

The sample archive database is managed by the Hubbard Brook USFS. Sample
metadata are processed from submission templates (L0) to harmonized
tables (L1), to appended files for accessions, collections, and samples
(L2). The workflow and products occurs in the USFS Pinyon (BOX)
environment. L2 files are staged with URL access for the Rshiny sample
search app.

\section{Templates}\label{templates}

Templates for submitting a new accession can be found \href{link}{here}

\section{Rshiny}\label{rshiny}

The interface for searching and downloading samples of interest is
developed in Rshiny. This app allows search at the collection level and
full sample search across all collections.

\bookmarksetup{startatroot}

\chapter{RAC Proposal Submissions}\label{rac-proposal-submissions}

\section{Overview}\label{overview-5}

The

\section{Access}\label{access}

\bookmarksetup{startatroot}

\chapter{Templates}\label{templates-1}

\section{Email}\label{email}

\begin{itemize}
\item
  Confirm addition to listserv, invite to have a profile page, introduce
  IM support

  Access \href{Email_Welcome2HBR.md}{Email template}
\item
  IM introduction following RAC proposal acceptance
\end{itemize}

\section{Data Submission Template}\label{data-submission-template}

\section{Sample Archive Submission
Template}\label{sample-archive-submission-template}

\bookmarksetup{startatroot}

\chapter{Computer Resources}\label{computer-resources}

\bookmarksetup{startatroot}

\chapter{IT Resources}\label{it-resources}

IT support for the UNH IM team is available through the UNH Research
Computer Center (RCC). RCC provides support to researchers in the
Institute for the Study of Earth, Oceans, and Space, as well as to the
wider University research community, State of NH, and Federal Agencies.
ESRC has had a long-standing Service Level Agreement (SLA) with RCC
(more than 20 years) which can be provided to reviewers upon request. In
addition to overall IT support described in the SLA, RCC also provides
the personnel for as-needed project support. This gives the HBR-IM team
access to expertise for special projects, without the need to provide
ongoing support for personnel on the IM team for programming, web
design, etc. HBR-IM has also made use of an \emph{RCC-funded} internship
program, wherein computer science undergraduates are paired with
researchers in the Institute for Earth, Oceans, and Space.

\section{Software used by IM}\label{software-used-by-im}

Table 1 outlines software in use by HBR-IM to manage data package
development and the Hubbard Brook website.

Table 1. Features of HBR Information Management System

\begin{longtable}[]{@{}
  >{\raggedright\arraybackslash}p{(\columnwidth - 2\tabcolsep) * \real{0.5139}}
  >{\raggedright\arraybackslash}p{(\columnwidth - 2\tabcolsep) * \real{0.4861}}@{}}
\toprule\noalign{}
\begin{minipage}[b]{\linewidth}\raggedright
\textbf{Feature}
\end{minipage} & \begin{minipage}[b]{\linewidth}\raggedright
\textbf{Details, software, resources}
\end{minipage} \\
\midrule\noalign{}
\endhead
\bottomrule\noalign{}
\endlastfoot
Website: \href{https://hubbardbrook.org/}{https://hubbardbrook.org} &
WordPress, html, css, php, xslt, javascript, apache, piwigo \\
Bibliography & Zotero, Zotpress wordpress plugin \\
Data Catalog & EDI data repository, local WordPress gateway to EDI HBR
data, EDIutils R \\
Metadata & ezEML, PostgreSQL, EML R package, EML2.1 \\
Computer Hardware & Dell Poweredge R510, desktop and laptop linux
systems. \\
Backup & BackupPC, rsnapshot, daily, weekly and monthly backups, on and
off-site \\
Data management & R, LibreOffice, QGIS, MySQL, PostgreSQL, git \\
\end{longtable}

\section{Account access}\label{account-access}

A number of accounts require access for HBR Information Management. In
some cases, access can be granted through the UNH Research Computing
Center, and in other cloud-based accounts the IM has designated a backup
person to have account access.

\begin{itemize}
\tightlist
\item
  IM desktop and server {[}IM/RCC{]}
\item
  UNH HB sharepoint {[}IM/Contosta at unh{]}
\item
  Zotero {[}IM/BU{]}
\item
  Databases {[}IM/RCC{]}
\item
  Piwigo {[}IM/RCC{]}
\item
  EDI/ezEML {[}IM/EDI{]}
\item
  Jotform {[}IM/Keeling at Cary{]}
\item
  Bluehost {[}IM/Post at 6288media{]}
\item
  GitHub
\item
  hubbardbrook gmail {[}IM/BU{]}
\end{itemize}

\section{Computer Resources}\label{computer-resources-1}

The Earth Systems Research Center's (ESRC) Science Computing Facility
(SCF) has a wide range of computer servers, printers, plotters,
archiving systems, software, data archives, and web based data
distribution systems that are integrated using several internal networks
and connected to the outside world through a high speed pipe. The
overall SCF administration is provided by the Research Computing Center
(RCC) located in the Institute for the Study of Earth, Oceans and Space
(EOS). Scientific data processing and analysis support is distributed
throughout workgroups within the center with additional centralized
expertise provided by ESRC's Laboratory for Remote Sensing and Spatial
Analysis. RCC's Lenharth Data Center was upgraded in September 2011.
This upgrade brought in new APC UPS units, energy efficient in-rack
cooling systems, and monitoring hardware to provide more space and
capability for future growth. Within this proposal, we take advantage of
this existing computer infrastructure, to meet our anticipated
computational needs.

The main ESRC servers consist of high-end, multi-processor computing
systems manufactured by Dell. The Dell systems run Linux and are used
for CPU intensive jobs, parallel modeling, and storage. Backups and
archives are done using BackupPC a disk-to-disk based system. Most of
the main servers share a gigabit (Gb) switch with the archive/backup
system for high-speed communications. All of this equipment is kept
within a physically secured, humidity and temperature controlled data
center complete with closed circuit video surveillance and a remotely
monitored security alarm system. Final data and image products are
produced from several ink-jet plotters and laser printers within the
department. Additionally, several CD/DVD writers are used for data
distribution.

EOS includes a CRAY XE6m-200 with 132 nodes, over 4000 processors and
160Tb of storage. In addition, our infrastructure has been strategically
upgraded to provide gigabit networking to desktops.

Individual scientists and research groups have additional computing
resources at their disposal. These include dedicated servers, individual
workstations, and various peripheral devices. The group servers and
individual workstations include: Linux servers and workstations, Windows
workstations, Apple Macintosh desktop and laptop computers. All servers,
user systems and networked peripheral devices are accessible within EOS
through a gigabit ethernet network. Wireless networking is available
throughout the building, including access to EduRoam. These systems also
have access to both Internet 1 and Internet 2.

ESRC currently houses a 65TB+ geographically referenced data archive
used for spatial data processing and analysis. This archive stored on
RAID5 data disks served by a series of data servers, houses global,
regional and local, Landsat, MODIS, IKONOS, Hyperion, ASTER, and SPOT
satellite imagery, land cover classified products, vegetation and other
indexes (EVI, LSWI, NDSI, NDVI, NDWI, LAI), aerial photography and GIS
vector data layers for use by all projects within the department.
Portions of these data, processed data products, and project results
archive are disseminated and distributed through several dozen regularly
updated and maintained ESRC operated websites. These websites are served
through a variety of web servers running Apache web server software
supported by other applications and libraries such as Tomcat, Web
Mapping Server (WMS), OpenLayers and other geographically enhanced
libraries such as GDAL, PROJ4, and GCTP.

ESRC also leverages the center's Laboratory for Remote Sensing and
Spatial Analysis, a spatial information processing, analysis and
distribution research laboratory. This laboratory provides geographic
information system (GIS), Web Mapping, spatial data archiving, data
distribution, remote sensing, image processing, cartography, large
format printing and scanning support to several ESRC and EOS research
projects. Staffed by professional geo-spatial information technicians,
computer programmers, and graduate and undergraduate university
students, the laboratory houses a multiple seat Linux, PC, and Mac OS
computer cluster supplied with a variety of open source Remote Sensing,
GIS, web mapping, image processing and cartography software and ESRI
ArcGIS, Leica ERDAS Imagine, and IDL/ENVI, commercial site, block, and
individually licensed GIS and Image processing software.

\bookmarksetup{startatroot}

\chapter*{References}\label{references}
\addcontentsline{toc}{chapter}{References}

\markboth{References}{References}

\phantomsection\label{refs}
\begin{CSLReferences}{0}{1}
\end{CSLReferences}




\end{document}
